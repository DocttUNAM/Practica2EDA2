\documentclass{report}
\usepackage[utf8]{inputenc}
\usepackage[Bjornstrup]{fncychap}
\usepackage{graphicx}
\usepackage{amsmath}
\usepackage{amssymb}
\usepackage[spanish]{babel}
\usepackage{nicematrix}
\usepackage{apacite}
\bibliographystyle{apacite}
\selectlanguage{spanish}
\addto\captionsspanish{\renewcommand{\abstractname}{Abstract}}

\begin{document}





  

  {
  \setlength{\columnsep}{4mm}
  \setlength{\parindent}{0.5in}
  \setlength{\parskip}{1em}
  \renewcommand{\baselinestretch}{1.5}
  \setlength{\headheight}{33pt}

  \thispagestyle{empty}
			\begin{figure}[ht]
				\minipage{0.87\textwidth}
					 \includegraphics[width=2cm]{fi.jpg}
					 \label{escudoFI}
				\endminipage
				\minipage{0.32\textwidth}
					 \includegraphics[height = 2.25cm ,width=2cm]{unam.png}
					 \label{EscuoUNAM}
				 \endminipage
					 %%\vspace{-1cm}
			 \end{figure}
			 
			 \vspace{0.1cm}
			 
			 \begin{center}
				 {\scshape\LARGE \textbf{Universidad Nacional Autónoma de México} \par}
				 {\scshape\Large Facultad de Ingeniería\par}
				 
				  {\LARGE ESTRUCTURAS DE DATOS Y ALGORITMOS II}
	 
				 % Restauramos el interlineado:
				 \begin{center}
				 
				 {\LARGE \textit{Grupo: 07 - Semestre: 2024-1}}
	 
				 

				 {\LARGE\bfseries PROYECTO 2 – ALGORITMOS DE BÚSQUEDA.\par}
	 
			 {\scshape\Large Fecha de entrega: 08/10/2023\par}	
	 
						 \LARGE	{ \textbf{Profesor:}}\\%% \textbf son negritas
			 \large		{ Edgar Tista Garcia}
			 
			 \vspace{-0.5cm}	
			 
			 \LARGE	{ \textbf{Alumno(s):}}\\%% \textbf son negritas
	 
			 \normalsize	 {Hernandez Gallardo Daniel Alonso}
			 
			 \vspace{-0.5cm}
			 
			 \normalsize		{Perez Osorio Luis Eduardo}
			 
			 \vspace{-0.5cm}
			 
			 \normalsize		{Valle Chavez Anton Yael}

			 
	 %% \it es letra itálica
					 \vspace{1.25cm}
					 \vspace{0.9cm}
					 
				 \end{center}
		 
			 \end{center} 
}



  \begin{abstract}
  This information presents a simple application of linear transformations in the study of synchronization of physical systems. To fully understand the presented information, certain theoretical knowledge in the field of linear transformations and synchronization of physical systems is required. Key concepts include linear transformations, their properties, the concept of kernel or null space, synchronization of systems, graphs, directed spanning trees, and properties of the Laplacian matrix. These concepts provide the necessary theoretical foundation to comprehend and analyze the application of linear transformations in the study of synchronization of physical systems.

The development section includes two activities. In Activity 1, a network of 6 robots described by Newton's second law is considered, and the graph and incidence matrix are presented. The Laplacian matrix is calculated, and its eigenvalues and kernel are determined. In Activity 2, a network of 8 robots is analyzed using similar steps. The graph, incidence matrix, Laplacian matrix, eigenvalues, and kernel are obtained. MATLAB simulations are performed to visualize the convergence of the nodes in both activities.

Overall, this information provides a practical demonstration of using linear transformations and graph theory concepts for studying the synchronization of physical systems.
  \end{abstract}
    

 
  \section*{Resumen}
    Esta información presenta una aplicación sencilla de las transformaciones lineales en el estudio de la sincronización de sistemas físicos. Para comprender completamente la información presentada, se requiere cierto conocimiento teórico en el campo de las transformaciones lineales y la sincronización de sistemas físicos. Los conceptos clave incluyen las transformaciones lineales, sus propiedades, el concepto de núcleo o espacio nulo, la sincronización de sistemas, los grafos, los árboles de expansión dirigidos y las propiedades de la matriz Laplaciana. Estos conceptos brindan los fundamentos teóricos necesarios para comprender y analizar la aplicación de las transformaciones lineales en el estudio de la sincronización de sistemas físicos.

La sección de desarrollo incluye dos actividades. En la Actividad 1, se considera una red de 6 robots descrita por la segunda ley de Newton, y se presenta el grafo y la matriz de incidencia. Se calcula la matriz Laplaciana y se determinan sus valores propios y núcleo. En la Actividad 2, se analiza una red de 8 robots utilizando pasos similares. Se obtienen el grafo, la matriz de incidencia, la matriz Laplaciana, los valores propios y el núcleo. Se realizan simulaciones en MATLAB para visualizar la convergencia de los nodos en ambas actividades.

En general, esta información proporciona una demostración práctica del uso de las transformaciones lineales y los conceptos de teoría de grafos para estudiar la sincronización de sistemas físicos.



\tableofcontents

  

\chapter{Introducción} 
    \section{Objetivo}
    Que el alumno desarrolle aplicaciones para la búsqueda por comparación de llaves y la
    transformación de llaves junto con la solución de colisiones
    \section{investigación}
      \subsection*{Framework}
\newpage
\chapter{Desarrollo}

\chapter{Conclusiones}
\textbf{Hernandez Gallardo Daniel Alonso} \\
\newpage
\textbf{Perez Osorio Luis Eduardo} \\
\newpage
\textbf{Valle Chavez Anton Yael} \\
\newpage
\nocite{*}
  \newpage
\bibliography{citas.bib}

\end{document}
